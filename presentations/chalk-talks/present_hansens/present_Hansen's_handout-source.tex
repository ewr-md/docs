\documentclass{article}
\usepackage[margin=1.0in]{geometry}
\usepackage{multicol} %unables ability to create columns of text

\usepackage{hyperref}
\hypersetup{
	colorlinks=true, %set true if you want colored links
	linktoc=all,     %set to all if you want both sections and subsections linked
	linkcolor=black,  %choose some color if you want links to stand out
}

\title{Neurologic Manifestations Hansen's Disease}
\author{Eric W. Robbins}
\date{}
\begin{document}
\maketitle
\tableofcontents
\newpage
\section{Why Care?}
\subsection{Prevalence}
As of 2015, a total of 13,950 HD cases were registered in the National Hansen’s Disease Registry in the United States since 1894, 178 of which were added in 2015.  While 57\% of the 2015 cases recorded a location other than the US as their place of birth, more than two-thirds of cases from Texas, Louisiana, Arkansas, Mississippi, and Florida were native-born US citizens with no residence history outside the US. Evidence suggests that zoonotic transmission from nine-banded armadillos is the principal source of infection perpetuating infection in these locales. 

Based on estimates of life expectancy, some 9,140 of total cases in the US are potentially still living.  The National Hansen’s Disease Registry is administered by the Health Resources and Services Administration, Healthcare Systems Bureau’s Division of National Hansen’s Disease Programs.  

In 2014, the World Health Organization reported that only 213,899 new cases were registered worldwide, representing a greater than 60 percent decline in annual new case numbers since 2001. 

\subsection{Morbidity/Costs}
\subsection{Massachusetts/Lahey}
The countries with the highest prevalence/incidence of Hansen's include, in order, 
\section{What is Hansen's?}
\subsection{Organism}
Hansen’s disease (also called leprosy) is a chronic infectious disease caused by the bacterium Mycobacterium leprae.  Humans are the only reservoir of proven significance for HD, however evidence suggests that leprosy in armadillos may be naturally transmitted to humans. The exact mechanism for the acquisition and transmission of HD is not known, however, it is not acquired from casual contact, such as shaking hands, sitting next to someone on a bus, or sitting together at a meal. HD is far less contagious than other infectious diseases. More than 95 percent of the human population has a natural immunity to the disease. Healthcare workers rarely contract HD. It is  characterized by the involvement primarily of skin, as well as peripheral nerves and the mucosa of the upper airway. Clinical forms of HD represent a spectrum reflecting the cellular immune response to Mycobacterium leprae. The following characteristics are typical of the major forms of the disease:
	\begin{itemize}
		\item Tuberculoid: one or a few well-demarcated, hypopigmented, and hypoesthetic or anesthetic skin lesions, frequently with active, spreading edges and a clearing center; peripheral nerve swelling or thickening also may occur
		\item Lepromatous: a number of erythematous papules and nodules or an infiltrate of the face, hands, and feet with lesions in a bilateral and symmetrical distribution that progress to thickening of the skin, possibly with reduced sensation.
		\item Borderline (dimorphous): skin lesions characteristic of both the tuberculoid and lepromatous forms
		\item Indeterminate: early lesions, usually hypopigmented macules, without developed tuberculoid or lepromatous features but with definite identification of acid-fast bacilli in biopsies.
	\end{itemize}
Borderline HD tends to shift toward the lepromatous form in the untreated patient and toward the tuberculoid form in the treated patient. Indeterminate leprosy is an early form that may develop into any of the other forms.
\subsection{Pathophysiology}
\section{Symptoms}
 Signs and symptoms may not appear for 2\textendash20 years.
\subsection{Paresthesias}
\subsection{ Analgeisic patches}
\subsection{Nerve palsies}
\section{Signs}
\subsection{Hypopigmentation}
\subsection{Analgeisic patches}
\subsection{Nerve palsies}
\subsection{Palpable nerves}
\section{Treatment}
\subsection{Antibiotics}
\subsection{Immunosuppresion}
In severe type 1 reactions (reversal reactions) w/ neuritis, first line is prednisone 40\textendash60 mg/day. Second line is cyclosporine, if not responding or unable to take GCCs. In pt w/ concomitant DM, MTX may be used as a steroid-sparing regimen.

In type 2 reactions (erythema nodosum leprosum), similar tx as in T1R. For ENL, can give thalidomide. Thalidomide is administered initially in a dose of 300 to 400 mg daily; frequently, this regimen controls the reaction within 48 hours. Subsequently, the dose should be tapered to a maintenance level, generally around 100 mg daily; every few months, attempts are made to taper off the drug.
\subsection{ENL}
\section{Refereneces}
	\bibliographystyle{unsrt}
	\bibliography{presenting-hansens}
	\cite{khardenavis2014}
	\cite{chandranesan2018}
	\cite{rao2016}
	\cite{ding2019}
	\cite{baveja2019}
	\cite{prasoon2015}
	\cite{shelley2018}
\end{document}