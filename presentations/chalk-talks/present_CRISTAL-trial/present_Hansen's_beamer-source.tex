\documentclass{beamer}
\usetheme{CambridgeUS}
\usecolortheme{dolphin}
%themes to test
%Theme to: AnnArbor Antibes Bergen Berkeley Berlin Boadilla boxes CambridgeUS Copenhagen Darmstadt default Dresden Frankfurt Goettingen Hannover Ilmenau JuanLesPins Luebeck Madrid Malmoe Marburg Montpellier PaloAlto Pittsburgh Rochester Singapore Szeged Warsaw
%Color to: albatross beaver beetle crane default dolphin dove fly lily orchid rose seagull seahorse sidebartab structure whale wolverine
%Font to: default professionalfonts serif structurebold structureitalicserif structuresmallcapsserif

%allows picture file insertion into document
\usepackage{graphicx}



\title{CRISTAL Trial}
\author{Eric W. Robbins}
\date{}
\begin{document}
	\begin{frame}
		\maketitle
	\end{frame}
	\begin{frame}
		\tableofcontents
	\end{frame}
\section{Background}
\subsection{Prior Trials}
	\begin{frame}
		\frametitle{Prior Trials}
		SAFE Trial, 2004 \emph{NEJM}. 4\% Albumin vs NS. 28-day mortality (doi: 10.1056/NEJMoa040232)
		Cochrane Review, 2013. Colloids vs crystalloids. Mortality (doi: 10.1002/14651858.CD000567.pub6)
		\
		ALBIOS Trial, 2014 \emph{NEJM}. 20\% Albumin vs usual care. 28-day mortality (doi: 10.1056/NEJMoa1305727)
	\end{frame}
\subsection{Study Question}
\begin{frame}
		\frametitle{PICO Format Question}
			\begin{itemize}
				\item P: ICU pt w/ hypovolemic shock
				\item I: colloid IVF resus
				\item C: crystalloid IV fluid resus 
				\item O: 28-day all-cause mortality
			\end{itemize}
	\end{frame}
\section{Trial Design}
\subsection{Trial Overview}
Multicenter, open-label, randomized comparative trial
N=2,857 ICU patients with hypovolemic shock
Colloids (n=1,414)
Crystalloids (n=1,443)
Setting: 57 ICUs in France, Belgium, Canada, Algeria, and Tunisia
Enrollment: 2003-2012 (stopped after interim analysis)
Follow-up: 90 days
Analysis: Intention-to-treat
Primary outcome: All-cause mortality at 28 days
\subsection{Inclusion Criteria}
\begin{frame}
	\frametitle{Inclusion Criteria}
Hypovolemic patients requiring fluid resuscitation who had not received fluid during the current hospital course.
Acute hypovolemia defined by:
MAP <60mmHg
Orthostatic hypotension (decrease in systolic >20mmHg from supine to semirecumbent)
Delta pulse pressure >13\%
Evidence of low filling pressures and low cardiac index (assessed invasively or noninvasively)
Signs of tissue hypoperfusion or hypoxia, including 2:
Glasgow Coma Scale score <12
Mottled skin
UOP <25mL/hr
Capillary refilling time >3 sec
Arterial lactate >2mmol/L
BUN >56mg/dL
Fractional excretion of sodium <1\%
\end{frame}
\subsection{Exclusion Criteria}
\begin{frame}
	\frametitle{Exclusion Criteria}
Previous fluid therapy in ICU
Anesthesia-related hypotension
Advanced chronic liver failure
Chronic renal failure
Acute anaphylaxis
Inherited coagulation disorder
DNR order
Pregnant
Burned >20\% of BSA
Allergy to any study drug
Refused consent
Dehydrated
Brain death/organ donor
	\end{frame}
\subsection{Interventions}
\begin{frame}
	\frametitle{Interventions}
Patients were randomized to a group with stratification by center and admission diagnoses
Colloids: Volume resuscitation with either hypooncotic solutions (eg, gelatins, 4-5\% albumin) or hyperoncotic solutions (eg, dextrans, hydroxyethylstarches, 20-25\% albumin)
Crystalloids: Volume resuscitation with isotonic saline, hypertonic saline, or buffered solutions (eg, LR)
Investigators chose which type of the predesignated fluid and how much to administer except:
Total dose of hydroxyethyl starch was maximum of 30 mL/kg
Must follow local regulatory agency recommendations
Both groups received maintenance fluid with isotonic crystalloids
Both groups received albumin for hypoalbuminemia (<2 g/L) per the discretion of the treating physician
Discontinuation of treatment on transfer out of ICU
\end{frame}
\subsection{Outcomes}
\begin{frame}
	\frametitle{Primary Outcome}
All-cause mortality at 28 days
25.4\% vs 27.0\% (RR 0.96; 95\% CI 0.88-1.04; P=0.26)
\end{frame}
	\begin{frame}
		\frametitle{Secondary Outcomes}
All-cause mortality at 90 days
30.7\% vs. 34.2\% (RR 0.92; 95\% CI 0.86-0.99; P=0.03; NNT=29)
Alive and not on RRT
First 7 days: 4.8 vs. 4.6 days (MD 0.2; 95\% CI -0.4 to 0.8; P=0.99)
First 28 days: 13.9 vs. 13.1 days (MD 0.8; 95\% CI -1.6 to 3.3; P=0.90)
Alive and not on mechanical ventilation
First 7 days: 2.1 vs. 1.8 days (MD 0.30; 95\% CI 0.09 to 0.48; P=0.01)
First 28 days: 14.6 vs. 13.5 days (MD 1.10; 95\% CI 0.14-2.06; P=0.01)
Alive and not on vasopressors
First 7 days: 5.0 vs. 4.7 days (MD 0.30; 95\% CI -0.03 to 0.50; P=0.04)
First 28 days: 16.2 vs. 15.2 days (MD 1.04; 95 \% CI -0.04 to 2.10; P=0.03)
All-cause mortality in ICU
25.1\% vs. 28.1\% (RR 0.92; 95\% CI 0.85-1.00; P=0.06)
All-cause mortality in hospital
30.1\% vs. 32.6\% (RR 0.94; 95\% CI 0.87-1.02; P=0.07)
ICU stay in first 28 days
8.3 vs. 8.1 days (MD 0.2; 95\% CI -0.5 to 0.9; P=0.69)
Hospital stay in first 28 days
11.9 vs. 11.6 days (MD 0.3; 95\% CI -0.5 to 1.1; P=0.37)
Any RRT
11.0\% vs. 12.5\% (RR 0.93; 95\% CI 0.83-1.03; P=0.19)
Median volume administered by day 7 in ICU
2L vs. 3L (P<0.001)
Duration of resuscitation therapy
2 vs. 2 days (P=0.93)
Receipt of albumin, crystalloid group
16.4\%
Resuscitation with colloid, crystalloid group
Gelatins: 1.7\%
Hydroxyethyl starch: 4.8\%
Resuscitation with crystalloid, colloid group
NS: 17.8\%
LR: 6.2\%
Hypertonic saline: 1.3\%
BP, UOP, weight, and CXR scores in first 24 hours
No difference
Blood product transfusion
No difference
	\end{frame}

	\begin{frame}
		\frametitle{Subgroup Analysis}
Primary outcome:
Sepsis: 27.8\% vs. 29.0\% (HR 0.95; 95\% CI 0.78-1.10)
Trauma: 15.3\% vs. 13.0\% (HR 0.93; 95\% CI 0.80-1.10)
Hypovolemic shock not from sepsis or trauma: 23.6\% vs. 26.6\% (HR 0.87; 95\% CI 0.69-1.10)
Interaction of homogeneity across the three strata P=0.07
	\end{frame}


\section{Criticisms}
Open label design
Long recruitment period, which didn't achieve enrollment goals due to early trial discontinuation
Not powered to detect the primary endpoint
Researchers estimated they would need 1,505 patients per group (total of 3,010 patients) to detect an absolute difference of 5\% in 28-day mortality with colloids
The trial hints that there may be a benefit of administering colloids to improve 90-day all-cause mortality
Clinicians were not blinded to fluid assignment because the researchers could not realistically stock enough blinded solutions in the ICUs
Recruitment and study period exceeded 9 years
Comparison of two classes of fluids and allowed clinicians to use whatever member of the class was available, not two specific fluid formulations
Initiation of renal replacement therapy may have been biased by physician knowledge of allocation (i.e. increased use or RRT in patients receiving colloids)
No adverse events reported
\end{document}
